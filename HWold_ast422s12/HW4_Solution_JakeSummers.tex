\documentclass[12pt, oneside, reqno]{amsart}
\usepackage{geometry}
\geometry{margin=1in}
\usepackage{graphicx}
\newcommand{\unit}[1]{\ensuremath{\, \mathrm{#1}}}
\usepackage{amssymb}
\graphicspath{ {downloads/} }

\begin{document}
\noindent \textbf{\underline{AST 322 - Introduction to Galactic and Extragalactic Astrophysics - HW 4}} \\

\fbox{\parbox{\textwidth}{\vskip.2em 
\textbf{4.1 (5 pts)} Derive from Equation 4.10, combined with Newton's 2nd Law, that Equation 4.12 is true:
\begin{equation*}
    \frac{1}{2}\left( \frac{dR_s}{dt} \right)^2 = \frac{GM_s}{R_s(t)}+U
\end{equation*}
Where $U$ is a constant of integration.
\vskip.2em }} 
\vskip1em 

\fbox{\parbox{\textwidth}{\vskip.2em 
\textbf{4.3 (7 pts)}\\ i) Show that Equation 4.28a is true (from 4.20):
\begin{equation*}
    H_0^2=\frac{8\pi G}{3c^2}\varepsilon_0 -\frac{\kappa c^2}{R_0^2}
\end{equation*}
ii) What is the special meaning of the case $\kappa = 0$? \\
iii) What is the actual value of the critical density of the universe $\rho_0$ that you derive in that case? i.e., show that Equation 4.28b is true:
\begin{equation*}
    \rho_0=\frac{3H_0^2}{8\pi G}
\end{equation*}
iv) Assuming the constants given in the text, calculate the numerical value of $\rho_o$ from 4.28b. [Do 4.3 before 4.2]
\vskip.2em }} 
\vskip1em 

\fbox{\parbox{\textwidth}{\vskip.2em 
\textbf{4.2 (8 pts)} \\
i) Derive Equation 4.17 from the equations that come before it:
\begin{equation*}
    \frac{1}{2}r_s^2\dot{a}^2=\frac{4\pi}{3}Gr_s^2\rho(t)a(t)^2+U
\end{equation*}
Then derive Equation 4.18 from that:
\begin{equation*}
    \left( \frac{\dot{a}}{a} \right)^2 = \frac{8\pi G}{3}\rho(t)+\frac{2U}{r_s^2}\frac{1}{a(t)^2}
\end{equation*}
ii) Solve Equation 4.18 for $a(t)$ in the case of $\kappa = 0$, i.e., $U=0$. Then sketch this solution. \\
iii) Discuss and sketch the solutions for the case of $\kappa=+1$ ($U<0$) and $\kappa=-1$ ($U>0$).
\vskip.2em }} 
\vskip1em 

\fbox{\parbox{\textwidth}{\vskip.2em 
\textbf{(EXTRA CREDIT)} \\
i) (1 pt) What is the end behavior of your $a(t)$ for the case of $\kappa=0$? What does this mean physically? Hint: calculate
\begin{equation*}
    \lim_{t\to\infty}\frac{da}{dt}
\end{equation*}
ii) (2 pts) Derive from Equation 4.18 that $a(t)$ has critical points (minima or maxima) determined by Equation 4.19:
\begin{equation*}
    a_{\text{max}} = -\frac{GM_s}{Ur_s}
\end{equation*}
Discuss what this means for the case of $\kappa>0$ and $\kappa<0$.
\vskip.2em }} 
\vskip1em 


\end{document}