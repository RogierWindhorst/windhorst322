%\documentclass[12pt]{article} %% required

%% style for page

\documentclass[letterpaper,12pt]{article}
\usepackage{epsfig,colortbl}
\usepackage{amssymb}
\textwidth=6.5in
\textheight=9.5in
\topmargin=-0.75in
\oddsidemargin=0.0in
\evensidemargin=0.0in
\thispagestyle{empty} % no page numbers

\usepackage{times}
\usepackage{psfig}

%%%%% Main Body %%%%%
\begin{document} %% required

{\Large \textbf{Homework 6}} \hfill 
{\small \emph{AST 422 Spring 2007}}\\
{\small .} \hfill {\small last updated: Mar 13, 2007}

\begin{itemize}
\item[(6.1.optional)] Evaluate equation 6.8 for a reasnable matirx of 
values of $\Omega_r$, $\Omega_m$, and $\Omega_{\Lambda}$. ($\kappa = 0$ 
only, OK)\\
(Can turn this into term project, as long as you turn it into a general 
paperwith discussion.)

\item[$<$choice$>$] \textbf{You can do either 6.2 or 6.3.  Not necessary 
to do both of them, but you may chose to do so for some extra points.}

\item[(6.2)] Show and discuss eq 6.17 \& 6.18.\\
How would you define $\theta$?\\
Hint: For $\theta \equiv \theta_1$, Try:
\vspace{-0.3cm}
\begin{itemize}
  \item[(a)] $\kappa = +1$\\
   $\cos\theta_0 = (2 - \Omega_0)/\Omega_0$\\
   $\cos\theta_1 = (z + \cos\theta_0)/(1 + z)$
  \item[(b)] $\kappa = -1$\\
   $\cosh\theta_0 = (2 - \Omega_0)/\Omega_0$\\
   $\cosh\theta_1 = (z + \cosh\theta_0)/(1 + z)$
 \end{itemize}

\item[(6.3)] Show and discuss eq 6.20 and 6.21

\item[(6.4.a)] Show eq 6.26 (Prefer you do 6.4.b instead...)

\item[(6.4.b)] Show eq 6.28 and 6.31 and compute age of universe for WMAP 
values:\\
 $\Omega_m = 0.27,\, \Omega_{\Lambda} = 0.73,\, H_0 = 71$

\item[(6.5)] Show eq 6.37.  (Assume eq 6.37) Then derive eq 6.38--6.40 
from this and discuss about eq 6.41 (Handout says 6.40, but Dr. 
Windhorst's textbook says 6.41...)

\end{itemize}


\end{document} %% required
