%\documentclass[12pt]{article} %% required

%% style for page
\documentclass[letterpaper,12pt]{article}
\usepackage{epsfig,colortbl}
\usepackage{amssymb}
\textwidth=6.5in
\textheight=9.5in
\topmargin=-0.75in
\oddsidemargin=0.0in
\evensidemargin=0.0in



\thispagestyle{empty} % no page numbers
\usepackage{times}
\begin{document} %% required


{\Large \textbf{Homework 1}} \hfill {\small \emph{AST 422 Spring 2007}}\\
\begin{itemize}
  \item[(a)] Assume power, $P(z)$ (= Luminosity, $L(z)$), and 
density $\rho(z)$ are all constant.
  \begin{itemize}
  \vspace{-0.3cm}
    \item Show for the Euclidian case: $N(>S) \propto S^{-3/2}$\\
where $S$ = flux.\\

    \vspace{-0.3cm}
    \item Disscuss what will change if your assumptions don't hold?\\
  \end{itemize}

  \item[(b)] Compute the Sky Brightness (in Radio).  Integrate!!
  \begin{itemize}
    \vspace{-0.3cm}
    \item Assume $N(>S) \propto S^{-\gamma}$.  For which values of $\gamma$
the sky brightness is infinite (or finite).\\
This is called the \emph{Olbers' Paradox}.\\
  \end{itemize}

%  \vspace{-0.3cm}
  \item[(c)] Using the same assumptions as in part (a):
  \begin{itemize}
    \vspace{-0.3cm}
    \item Show for the Euclidian case: $N(<m) \propto 10^{0.6m}$\\
where, $m$ = magnitude.\\

    \vspace{-0.3cm}
    \item For $N(<m) \propto 10^{\alpha m}$, which values of $\alpha$ will 
cause the Olber's Paradox.\\

    \vspace{-0.3cm}
    \item Discuss about what you can conclude from the assumptions we made.

  \end{itemize}
\end{itemize}


\end{document} %% required
